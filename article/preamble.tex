% Graphics
\usepackage{xcolor}
\usepackage{graphicx}
\usepackage{tikz}
\usetikzlibrary{
	arrows, arrows.meta, positioning, fit
}
\usepackage{caption}
\usepackage{subcaption}  % grids of images
\usepackage[section]{placeins}  % figures stay in section

% Mathematics and code
\usepackage{amsthm,amsmath, amssymb}
\usepackage{mathtools}
\usepackage{flexisym}  % for breqn below
\usepackage{breqn}  % breaking long equations, dmath environment
\usepackage{mathrsfs}  % \mathscr command
\usepackage{algorithm}
\usepackage{algpseudocode}
\usepackage{thmtools}  % theorem, definition, and question lists

\theoremstyle{plain}
\newtheorem{theorem}{Theorem}[section]
\newtheorem{proposition}[theorem]{Proposition}
\newtheorem{lemma}[theorem]{Lemma}  
\newtheorem*{corollary}{Corollary}
\theoremstyle{definition}
\newtheorem{definition}{Definition}[section]
\newtheorem{example}{Example}[section]
\newtheorem{question}{Question}
\theoremstyle{remark}
\newtheorem*{remark}{Remark}

% Fonts, symbols, languages
\usepackage[utf8]{inputenc}
\usepackage{enumerate}  % Custom numbered items, such as steps 1, 2, etc.
\usepackage{enumitem}  % Ordinals
\usepackage{moreenum}
\usepackage{verbatim}

% References and links
\usepackage{hyperref}
\hypersetup{
    colorlinks=true,
    linkcolor=blue,
    filecolor=magenta,
    urlcolor=cyan
    }
\usepackage{url}

\endinput
